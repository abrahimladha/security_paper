\documentclass{beamer}
\usetheme{default}
 \usepackage{amsmath}
\title{Bitcoin}
\author{Abrahim Ladha}
\date{\today}


\begin{document}
\begin{frame}
\titlepage
\end{frame}

\begin{frame}
{\Huge What is bitcoin?}
Bitcoin is a cryptographically secured decentralized currency system.\\
Transactions are sent user to user without a central authority verification.
Transactions between users involve units of BTC and are verified on the blockchain, which is a list of all transactions ever made by all users. \\
a transaction cannot be undone with out first undoing every transaction that comes after it on the blockchain.

\end{frame}

\begin{frame}
{\Huge How are coins generated and distributed?}
Mining.
A miner does "work" on a block of bitcoins, which is basically just bruteforcing a long SHA256 password.\\
Once guessed correctly the miner receives those coins. \\
In practice users form "pools" and work together to solve a block.\\
Every so often a "halvening" occurs and the block reward is halved.\\
There can only ever exist 21 Million BTC.

\end{frame}

\begin{frame}

\includegraphics[scale=.3]{../../2016-03-08-013647_846x356_scrot.png}
\begin{figure}[ht!]
\includegraphics[scale=.15]{../../Downloads/1380373_10202467135509126_955771813059849142_n.jpg} 
\end{figure}
\end{frame}

\begin{frame}
{\Huge security?}
$p = $ probability honest node finds the next block\\
$q = $ probability attacker finds the next block\\
$q_z = $ probability the attacker will ever catch up from $z$ blocks behind \\
$
q_z =
\begin{cases}
1 & p \leq q \\
(q/p)^z &  p > q
\end{cases}
$\\
Let $\lambda = z\frac{q}{p}$.
Probability attack could catch up now is Poisson density for each amount of progress times the probability of catching up from that point:
$\sum_{k = 0}^{\infty} \frac{\lambda^k e^{-\lambda}}{k!} \times 
\begin{cases}
(\frac{q}{p})^{(z-k)} & k \leq z \\
1 &  k > z
\end{cases}$
\end{frame}

\begin{frame}
{\Huge Security continued}
Simplifying to $1 - \sum_{k = 0}^{\infty} \frac{\lambda^k e^{-\lambda}}{k!} (1- (\frac{q}{p})^{(z-k)})$\\
The probability drops off exponentially with z:\\
q=0.3\\
z=0    P=1.0000000\\
z=5    P=0.1773523\\
z=10   P=0.0416605\\
z=15   P=0.0101008\\
z=20   P=0.0024804\\
z=25   P=0.0006132\\
z=30   P=0.0001522\\
z=35   P=0.0000379\\
z=40   P=0.0000095\\
z=45   P=0.0000024\\
z=50   P=0.0000006\\
However, if $q \geq 0.5$, an attacker has enough power to undo the blockchain. but what are the odds $ 51\%$ of users are mailcious?
\end{frame}

\begin{frame}
Satoshi Nakamoto, Bitcoin: A Peer-to-Peer Electronic Cash System, https://bitcoin.org/bitcoin.pdf
\end{frame}


\end{document}